%%
% Copyright (c) 2017 - 2019, Pascal Wagler;
% Copyright (c) 2014 - 2019, John MacFarlane
%
% All rights reserved.
%
% Redistribution and use in source and binary forms, with or without
% modification, are permitted provided that the following conditions
% are met:
%
% - Redistributions of source code must retain the above copyright
% notice, this list of conditions and the following disclaimer.
%
% - Redistributions in binary form must reproduce the above copyright
% notice, this list of conditions and the following disclaimer in the
% documentation and/or other materials provided with the distribution.
%
% - Neither the name of John MacFarlane nor the names of other
% contributors may be used to endorse or promote products derived
% from this software without specific prior written permission.
%
% THIS SOFTWARE IS PROVIDED BY THE COPYRIGHT HOLDERS AND CONTRIBUTORS
% "AS IS" AND ANY EXPRESS OR IMPLIED WARRANTIES, INCLUDING, BUT NOT
% LIMITED TO, THE IMPLIED WARRANTIES OF MERCHANTABILITY AND FITNESS
% FOR A PARTICULAR PURPOSE ARE DISCLAIMED. IN NO EVENT SHALL THE
% COPYRIGHT OWNER OR CONTRIBUTORS BE LIABLE FOR ANY DIRECT, INDIRECT,
% INCIDENTAL, SPECIAL, EXEMPLARY, OR CONSEQUENTIAL DAMAGES (INCLUDING,
% BUT NOT LIMITED TO, PROCUREMENT OF SUBSTITUTE GOODS OR SERVICES;
% LOSS OF USE, DATA, OR PROFITS; OR BUSINESS INTERRUPTION) HOWEVER
% CAUSED AND ON ANY THEORY OF LIABILITY, WHETHER IN CONTRACT, STRICT
% LIABILITY, OR TORT (INCLUDING NEGLIGENCE OR OTHERWISE) ARISING IN
% ANY WAY OUT OF THE USE OF THIS SOFTWARE, EVEN IF ADVISED OF THE
% POSSIBILITY OF SUCH DAMAGE.
%%

%%
% For usage information and examples visit the GitHub page of this template:
% https://github.com/Wandmalfarbe/pandoc-latex-template
%%

\PassOptionsToPackage{unicode=true}{hyperref} % options for packages loaded elsewhere
\PassOptionsToPackage{hyphens}{url}
\PassOptionsToPackage{dvipsnames,svgnames*,x11names*,table}{xcolor}
%
\documentclass[
  12pt,
  a4paper,
  oneside,tablecaptionabove
]{scrbook}
\usepackage{lmodern}
\usepackage{setspace}
\setstretch{1.2}
\usepackage{amssymb,amsmath}
\usepackage{ifxetex,ifluatex}
\ifnum 0\ifxetex 1\fi\ifluatex 1\fi=0 % if pdftex
  \usepackage[T1]{fontenc}
  \usepackage[utf8]{inputenc}
  \usepackage{textcomp} % provides euro and other symbols
\else % if luatex or xelatex
  \usepackage{unicode-math}
  \defaultfontfeatures{Scale=MatchLowercase}
  \defaultfontfeatures[\rmfamily]{Ligatures=TeX,Scale=1}
\fi
% use upquote if available, for straight quotes in verbatim environments
\IfFileExists{upquote.sty}{\usepackage{upquote}}{}
\IfFileExists{microtype.sty}{% use microtype if available
  \usepackage[]{microtype}
  \UseMicrotypeSet[protrusion]{basicmath} % disable protrusion for tt fonts
}{}
\makeatletter
\@ifundefined{KOMAClassName}{% if non-KOMA class
  \IfFileExists{parskip.sty}{%
    \usepackage{parskip}
  }{% else
    \setlength{\parindent}{0pt}
    \setlength{\parskip}{6pt plus 2pt minus 1pt}}
}{% if KOMA class
  \KOMAoptions{parskip=half}}
\makeatother
\usepackage{xcolor}
\definecolor{default-linkcolor}{HTML}{A50000}
\definecolor{default-filecolor}{HTML}{A50000}
\definecolor{default-citecolor}{HTML}{4077C0}
\definecolor{default-urlcolor}{HTML}{4077C0}
\IfFileExists{xurl.sty}{\usepackage{xurl}}{} % add URL line breaks if available
\IfFileExists{bookmark.sty}{\usepackage{bookmark}}{\usepackage{hyperref}}
\hypersetup{
  pdftitle={Logbuch},
  pdfauthor={bertolino},
  pdfkeywords={Markdown},
  colorlinks=true,
  linkcolor=default-linkcolor,
  filecolor=default-filecolor,
  citecolor=default-citecolor,
  urlcolor=default-urlcolor,
  breaklinks=true}
\urlstyle{same}  % don't use monospace font for urls
\usepackage[margin=2.5cm,includehead=true,includefoot=true,centering]{geometry}
\usepackage{listings}
\newcommand{\passthrough}[1]{#1}
\lstset{defaultdialect=[5.3]Lua}
\lstset{defaultdialect=[x86masm]Assembler}
\usepackage{longtable,booktabs}
% Allow footnotes in longtable head/foot
\IfFileExists{footnotehyper.sty}{\usepackage{footnotehyper}}{\usepackage{footnote}}
\makesavenoteenv{longtable}
\usepackage{graphicx,grffile}
\makeatletter
\def\maxwidth{\ifdim\Gin@nat@width>\linewidth\linewidth\else\Gin@nat@width\fi}
\def\maxheight{\ifdim\Gin@nat@height>\textheight\textheight\else\Gin@nat@height\fi}
\makeatother
% Scale images if necessary, so that they will not overflow the page
% margins by default, and it is still possible to overwrite the defaults
% using explicit options in \includegraphics[width, height, ...]{}
\setkeys{Gin}{width=\maxwidth,height=\maxheight,keepaspectratio}
% Make links footnotes instead of hotlinks:
\DeclareRobustCommand{\href}[2]{#2\footnote{\url{#1}}}
\setlength{\emergencystretch}{3em}  % prevent overfull lines
\providecommand{\tightlist}{%
  \setlength{\itemsep}{0pt}\setlength{\parskip}{0pt}}
\setcounter{secnumdepth}{2}
% Redefines (sub)paragraphs to behave more like sections
\ifx\paragraph\undefined\else
  \let\oldparagraph\paragraph
  \renewcommand{\paragraph}[1]{\oldparagraph{#1}\mbox{}}
\fi
\ifx\subparagraph\undefined\else
  \let\oldsubparagraph\subparagraph
  \renewcommand{\subparagraph}[1]{\oldsubparagraph{#1}\mbox{}}
\fi

% Make use of float-package and set default placement for figures to H
\usepackage{float}
\floatplacement{figure}{H}

\renewcommand{\thetable}{\arabic{table}}
\renewcommand{\thefigure}{\arabic{figure}}
\renewcommand{\thesection}{\arabic{section}}
\usepackage{lipsum}
\usepackage[]{biblatex}
\addbibresource{reference.bib}

\title{Logbuch}
\usepackage{etoolbox}
\makeatletter
\providecommand{\subtitle}[1]{% add subtitle to \maketitle
  \apptocmd{\@title}{\par {\large #1 \par}}{}{}
}
\makeatother
\subtitle{Research/Academia life notes}
\author{bertolino}
\date{\today}





%%
%% added
%%

%
% No language specified? take American English.
%

\ifnum 0\ifxetex 1\fi\ifluatex 1\fi=0 % if pdftex
  \usepackage[shorthands=off,main=english]{babel}
\else
    % See issue https://github.com/reutenauer/polyglossia/issues/127
  \renewcommand*\familydefault{\sfdefault}
    % load polyglossia as late as possible as it *could* call bidi if RTL lang (e.g. Hebrew or Arabic)
  \usepackage{polyglossia}
  \setmainlanguage[]{english}
\fi


%
% colors
%
\usepackage[]{xcolor}

%
% listing colors
%
\definecolor{listing-background}{HTML}{F7F7F7}
\definecolor{listing-rule}{HTML}{B3B2B3}
\definecolor{listing-numbers}{HTML}{B3B2B3}
\definecolor{listing-text-color}{HTML}{000000}
\definecolor{listing-keyword}{HTML}{435489}
\definecolor{listing-identifier}{HTML}{435489}
\definecolor{listing-string}{HTML}{00999A}
\definecolor{listing-comment}{HTML}{8E8E8E}
\definecolor{listing-javadoc-comment}{HTML}{006CA9}

%\definecolor{listing-background}{rgb}{0.97,0.97,0.97}
%\definecolor{listing-rule}{HTML}{B3B2B3}
%\definecolor{listing-numbers}{HTML}{B3B2B3}
%\definecolor{listing-text-color}{HTML}{000000}
%\definecolor{listing-keyword}{HTML}{D8006B}
%\definecolor{listing-identifier}{HTML}{000000}
%\definecolor{listing-string}{HTML}{006CA9}
%\definecolor{listing-comment}{rgb}{0.25,0.5,0.35}
%\definecolor{listing-javadoc-comment}{HTML}{006CA9}

%
% for the background color of the title page
%
\usepackage{pagecolor}
\usepackage{afterpage}

%
% TOC depth and
% section numbering depth
%
\setcounter{tocdepth}{3}
\setcounter{secnumdepth}{3}

%
% break urls
%
\PassOptionsToPackage{hyphens}{url}

%
% When using babel or polyglossia with biblatex, loading csquotes is recommended
% to ensure that quoted texts are typeset according to the rules of your main language.
%
\usepackage{csquotes}

%
% captions
%
\definecolor{caption-color}{HTML}{777777}
\usepackage[font={stretch=1.2}, textfont={color=caption-color}, position=top, skip=4mm, labelfont=bf, singlelinecheck=false, justification=raggedright]{caption}
\setcapindent{0em}
\captionsetup[longtable]{position=above}

%
% blockquote
%
\definecolor{blockquote-border}{RGB}{221,221,221}
\definecolor{blockquote-text}{RGB}{119,119,119}
\usepackage{mdframed}
\newmdenv[rightline=false,bottomline=false,topline=false,linewidth=3pt,linecolor=blockquote-border,skipabove=\parskip]{customblockquote}
\renewenvironment{quote}{\begin{customblockquote}\list{}{\rightmargin=0em\leftmargin=0em}%
\item\relax\color{blockquote-text}\ignorespaces}{\unskip\unskip\endlist\end{customblockquote}}

%
% Source Sans Pro as the de­fault font fam­ily
% Source Code Pro for monospace text
%
% 'default' option sets the default
% font family to Source Sans Pro, not \sfdefault.
%
\usepackage[default]{sourcesanspro}
\usepackage{sourcecodepro}

% XeLaTeX specific adjustments for straight quotes: https://tex.stackexchange.com/a/354887
% This issue is already fixed (see https://github.com/silkeh/latex-sourcecodepro/pull/5) but the
% fix is still unreleased.
% TODO: Remove this workaround when the new version of sourcecodepro is reelased on CTAN.
\ifxetex
\makeatletter
\defaultfontfeatures[\ttfamily]
  { Numbers   = \sourcecodepro@figurestyle,
    Scale     = \SourceCodePro@scale,
    Extension = .otf }
\setmonofont
  [ UprightFont    = *-\sourcecodepro@regstyle,
    ItalicFont     = *-\sourcecodepro@regstyle It,
    BoldFont       = *-\sourcecodepro@boldstyle,
    BoldItalicFont = *-\sourcecodepro@boldstyle It ]
  {SourceCodePro}
\makeatother
\fi

%
% heading color
%
\definecolor{heading-color}{RGB}{40,40,40}
\addtokomafont{section}{\color{heading-color}}
% When using the classes report, scrreprt, book,
% scrbook or memoir, uncomment the following line.
\addtokomafont{chapter}{\color{heading-color}}

%
% variables for title and author
%
\usepackage{titling}
\title{Logbuch}
\author{bertolino}

%
% tables
%

\definecolor{table-row-color}{HTML}{F5F5F5}
\definecolor{table-rule-color}{HTML}{999999}

%\arrayrulecolor{black!40}
\arrayrulecolor{table-rule-color}     % color of \toprule, \midrule, \bottomrule
\setlength\heavyrulewidth{0.3ex}      % thickness of \toprule, \bottomrule
\renewcommand{\arraystretch}{1.3}     % spacing (padding)

% Reset rownum counter so that each table
% starts with the same row colors.
% https://tex.stackexchange.com/questions/170637/restarting-rowcolors
\let\oldlongtable\longtable
\let\endoldlongtable\endlongtable
\renewenvironment{longtable}{
\rowcolors{3}{}{table-row-color!100}  % row color
\oldlongtable} {
\endoldlongtable
\global\rownum=0\relax}

% Unfortunately the colored cells extend beyond the edge of the
% table because pandoc uses @-expressions (@{}) like so:
%
% \begin{longtable}[]{@{}ll@{}}
% \end{longtable}
%
% https://en.wikibooks.org/wiki/LaTeX/Tables#.40-expressions

%
% remove paragraph indention
%
\setlength{\parindent}{0pt}
\setlength{\parskip}{6pt plus 2pt minus 1pt}
\setlength{\emergencystretch}{3em}  % prevent overfull lines

%
%
% Listings
%
%

\lstdefinestyle{eisvogel_listing_style}{
  language         = java,
  numbers          = left,
  xleftmargin      = 2.7em,
  framexleftmargin = 2.5em,
  backgroundcolor  = \color{listing-background},
  basicstyle       = \color{listing-text-color}\small\ttfamily{}\linespread{1.15}, % print whole listing small
  breaklines       = true,
  frame            = single,
  framesep         = 0.6mm,
  rulecolor        = \color{listing-rule},
  frameround       = ffff,
  tabsize          = 4,
  numberstyle      = \color{listing-numbers},
  aboveskip        = 1.0em,
  belowcaptionskip = 1.0em,
  keywordstyle     = \color{listing-keyword}\bfseries,
  classoffset      = 0,
  sensitive        = true,
  identifierstyle  = \color{listing-identifier},
  commentstyle     = \color{listing-comment},
  morecomment      = [s][\color{listing-javadoc-comment}]{/**}{*/},
  stringstyle      = \color{listing-string},
  showstringspaces = false,
  escapeinside     = {/*@}{@*/}, % Allow LaTeX inside these special comments
  literate         =
  {á}{{\'a}}1 {é}{{\'e}}1 {í}{{\'i}}1 {ó}{{\'o}}1 {ú}{{\'u}}1
  {Á}{{\'A}}1 {É}{{\'E}}1 {Í}{{\'I}}1 {Ó}{{\'O}}1 {Ú}{{\'U}}1
  {à}{{\`a}}1 {è}{{\'e}}1 {ì}{{\`i}}1 {ò}{{\`o}}1 {ù}{{\`u}}1
  {À}{{\`A}}1 {È}{{\'E}}1 {Ì}{{\`I}}1 {Ò}{{\`O}}1 {Ù}{{\`U}}1
  {ä}{{\"a}}1 {ë}{{\"e}}1 {ï}{{\"i}}1 {ö}{{\"o}}1 {ü}{{\"u}}1
  {Ä}{{\"A}}1 {Ë}{{\"E}}1 {Ï}{{\"I}}1 {Ö}{{\"O}}1 {Ü}{{\"U}}1
  {â}{{\^a}}1 {ê}{{\^e}}1 {î}{{\^i}}1 {ô}{{\^o}}1 {û}{{\^u}}1
  {Â}{{\^A}}1 {Ê}{{\^E}}1 {Î}{{\^I}}1 {Ô}{{\^O}}1 {Û}{{\^U}}1
  {œ}{{\oe}}1 {Œ}{{\OE}}1 {æ}{{\ae}}1 {Æ}{{\AE}}1 {ß}{{\ss}}1
  {ç}{{\c c}}1 {Ç}{{\c C}}1 {ø}{{\o}}1 {å}{{\r a}}1 {Å}{{\r A}}1
  {€}{{\EUR}}1 {£}{{\pounds}}1 {«}{{\guillemotleft}}1
  {»}{{\guillemotright}}1 {ñ}{{\~n}}1 {Ñ}{{\~N}}1 {¿}{{?`}}1
  {…}{{\ldots}}1 {≥}{{>=}}1 {≤}{{<=}}1 {„}{{\glqq}}1 {“}{{\grqq}}1
  {”}{{''}}1
}
\lstset{style=eisvogel_listing_style}

\lstdefinelanguage{XML}{
  morestring      = [b]",
  moredelim       = [s][\bfseries\color{listing-keyword}]{<}{\ },
  moredelim       = [s][\bfseries\color{listing-keyword}]{</}{>},
  moredelim       = [l][\bfseries\color{listing-keyword}]{/>},
  moredelim       = [l][\bfseries\color{listing-keyword}]{>},
  morecomment     = [s]{<?}{?>},
  morecomment     = [s]{<!--}{-->},
  commentstyle    = \color{listing-comment},
  stringstyle     = \color{listing-string},
  identifierstyle = \color{listing-identifier}
}

%
% header and footer
%
\usepackage{fancyhdr}

\def \logbuchprojectmark {}
\def \logbuchsubjectmark {}
\def \logbuchsubjectdatemark {}
\def \logbuchsubjectauthmark {}

\fancypagestyle{eisvogel-header-footer}{
  \fancyhead[LO,RE]{\logbuchprojectmark}
  \fancyhead[C]{\logbuchsubjectmark}
  \fancyhead[RO,LE]{\logbuchsubjectdatemark}
  \fancyfoot[LO,RE]{\logbuchsubjectauthmark}
  \fancyfoot[C]{}
  \fancyfoot[RO,LE]{\thepage}
  \renewcommand{\headrulewidth}{0.4pt}
  \renewcommand{\footrulewidth}{0.4pt}
}
\pagestyle{eisvogel-header-footer}

%%
%% end added
%%

%%
%% Logbuch adition
%%

% \definecolor{partpage-color}{HTML}{}

% \setkomafont{chapter}{\normalfont\huge\sffamily\bfseries\color{blue}}
% \addtokomafont{section}{\color{cyan}}
\usepackage{color}
\usepackage{fancyvrb}
\newcommand{\VerbBar}{|}
\newcommand{\VERB}{\Verb[commandchars=\\\{\}]}
\DefineVerbatimEnvironment{Highlighting}{Verbatim}{commandchars=\\\{\}}
% Add ',fontsize=\small' for more characters per line
\usepackage{framed}
\definecolor{shadecolor}{RGB}{248,248,248}
\newenvironment{Shaded}{\begin{snugshade}}{\end{snugshade}}
\newcommand{\KeywordTok}[1]{\textcolor[rgb]{0.13,0.29,0.53}{\textbf{#1}}}
\newcommand{\DataTypeTok}[1]{\textcolor[rgb]{0.13,0.29,0.53}{#1}}
\newcommand{\DecValTok}[1]{\textcolor[rgb]{0.00,0.00,0.81}{#1}}
\newcommand{\BaseNTok}[1]{\textcolor[rgb]{0.00,0.00,0.81}{#1}}
\newcommand{\FloatTok}[1]{\textcolor[rgb]{0.00,0.00,0.81}{#1}}
\newcommand{\ConstantTok}[1]{\textcolor[rgb]{0.00,0.00,0.00}{#1}}
\newcommand{\CharTok}[1]{\textcolor[rgb]{0.31,0.60,0.02}{#1}}
\newcommand{\SpecialCharTok}[1]{\textcolor[rgb]{0.00,0.00,0.00}{#1}}
\newcommand{\StringTok}[1]{\textcolor[rgb]{0.31,0.60,0.02}{#1}}
\newcommand{\VerbatimStringTok}[1]{\textcolor[rgb]{0.31,0.60,0.02}{#1}}
\newcommand{\SpecialStringTok}[1]{\textcolor[rgb]{0.31,0.60,0.02}{#1}}
\newcommand{\ImportTok}[1]{#1}
\newcommand{\CommentTok}[1]{\textcolor[rgb]{0.56,0.35,0.01}{\textit{#1}}}
\newcommand{\DocumentationTok}[1]{\textcolor[rgb]{0.56,0.35,0.01}{\textbf{\textit{#1}}}}
\newcommand{\AnnotationTok}[1]{\textcolor[rgb]{0.56,0.35,0.01}{\textbf{\textit{#1}}}}
\newcommand{\CommentVarTok}[1]{\textcolor[rgb]{0.56,0.35,0.01}{\textbf{\textit{#1}}}}
\newcommand{\OtherTok}[1]{\textcolor[rgb]{0.56,0.35,0.01}{#1}}
\newcommand{\FunctionTok}[1]{\textcolor[rgb]{0.00,0.00,0.00}{#1}}
\newcommand{\VariableTok}[1]{\textcolor[rgb]{0.00,0.00,0.00}{#1}}
\newcommand{\ControlFlowTok}[1]{\textcolor[rgb]{0.13,0.29,0.53}{\textbf{#1}}}
\newcommand{\OperatorTok}[1]{\textcolor[rgb]{0.81,0.36,0.00}{\textbf{#1}}}
\newcommand{\BuiltInTok}[1]{#1}
\newcommand{\ExtensionTok}[1]{#1}
\newcommand{\PreprocessorTok}[1]{\textcolor[rgb]{0.56,0.35,0.01}{\textit{#1}}}
\newcommand{\AttributeTok}[1]{\textcolor[rgb]{0.77,0.63,0.00}{#1}}
\newcommand{\RegionMarkerTok}[1]{#1}
\newcommand{\InformationTok}[1]{\textcolor[rgb]{0.56,0.35,0.01}{\textbf{\textit{#1}}}}
\newcommand{\WarningTok}[1]{\textcolor[rgb]{0.56,0.35,0.01}{\textbf{\textit{#1}}}}
\newcommand{\AlertTok}[1]{\textcolor[rgb]{0.94,0.16,0.16}{#1}}
\newcommand{\ErrorTok}[1]{\textcolor[rgb]{0.64,0.00,0.00}{\textbf{#1}}}
\newcommand{\NormalTok}[1]{#1}


\begin{document}
%%
%% begin titlepage
%%

\begin{titlepage}
\newgeometry{left=6cm}
\definecolor{titlepage-color}{HTML}{3E3E3B}
\newpagecolor{titlepage-color}\afterpage{\restorepagecolor}
\newcommand{\colorRule}[3][black]{\textcolor[HTML]{#1}{\rule{#2}{#3}}}
\begin{flushleft}
\noindent
\\[-1em]
\color[HTML]{EBECED}
\makebox[0pt][l]{\colorRule[A2AAB0]{1.3\textwidth}{4pt}}
\par
\noindent

{ \setstretch{1.4}
\vfill
\noindent {\huge \textbf{\textsf{Logbuch}}}
\vskip 1em
{\Large \textsf{Research/Academia life notes}}
\vskip 2em
\noindent
{\Large \textsf{bertolino}
\vfill
}


\textsf{\today}}
\end{flushleft}
\end{titlepage}
\restoregeometry

%%
%% end titlepage
%%


\setcounter{chapter}{1}
\addtocounter{chapter}{-1}

\renewcommand*\contentsname{Projects}
{
\hypersetup{linkcolor=}
\setcounter{tocdepth}{3}
\pagestyle{plain}
\tableofcontents
\newpage
}

\part*{A repeated project \\ \bigskip\bigskip\bigskip\bigskip\bigskip\normalfont\large{bertolino}}
\addcontentsline{toc}{part}{A repeated project}

\def \logbuchprojectmark {A repeated project}

\chapter*{\LARGE{A copy from other subject}}

\addcontentsline{toc}{chapter}{$\circ$ A copy from other subject}
\setcounter{section}{0}

\def \logbuchsubjectmark {A copy from other subject}
\def \logbuchsubjectdatemark {09.03.2019 15:14}
\def \logbuchsubjectauthmark {\textit{bertolino}}
\thispagestyle{eisvogel-header-footer}
\textbf{The content below was copied from
http://www.unexpected-vortices.com/sw/rippledoc/quick-markdown-example.html
at 2019-03-09 and has some modifications.}

\hypertarget{an-h1-header}{%
\section{An h1 header}\label{an-h1-header}}

Paragraphs are separated by a blank line.

2nd paragraph. \emph{Italic}, \textbf{bold}, and
\passthrough{\lstinline!monospace!}. Itemized lists look like:

\begin{itemize}
\tightlist
\item
  this one
\item
  that one
\item
  the other one
\end{itemize}

Note that --- not considering the asterisk --- the actual text content
starts at 4-columns in.

\begin{quote}
Block quotes are written like so.

They can span multiple paragraphs, if you like.
\end{quote}

Use 3 dashes for an em-dash. Use 2 dashes for ranges (ex., ``it's all in
chapters 12--14''). Three dots \ldots{} will be converted to an
ellipsis. Unicode is supported. \textbf{\emph{It depends on the output
format. You may face problems while converting an unicode character to
LaTeX. For example, if you use the ``+emoji'' extension, pandoc will
generate unicode.}}

\hypertarget{an-h2-header}{%
\subsection{An h2 header}\label{an-h2-header}}

Here's a numbered list:

\begin{enumerate}
\def\labelenumi{\arabic{enumi}.}
\tightlist
\item
  first item
\item
  second item
\item
  third item
\end{enumerate}

Note again how the actual text starts at 4 columns in (4 characters from
the left side). Here's a code sample:

\begin{lstlisting}
# Let me re-iterate ...
for i in 1 .. 10 { do-something(i) }
\end{lstlisting}

As you probably guessed, indented 4 spaces. By the way, instead of
indenting the block, you can use delimited blocks, if you like:

\begin{lstlisting}
define foobar() {
    print "Welcome to flavor country!";
An h1 header
============

Paragraphs are separated by a blank line.

2nd paragraph. *Italic*, **bold**, and `monospace`. Itemized lists
look like:

  * this one
  * that one
  * the other one

Note that --- not considering the asterisk --- the actual text
content starts at 4-columns in.

> Block quotes are
> written like so.
>
> They can span multiple paragraphs,
> if you like.

Use 3 dashes for an em-dash. Use 2 dashes for ranges (ex., "it's all
in chapters 12--14"). Three dots ... will be converted to an ellipsis.
Unicode is supported.



An h2 header
------------

Here's a numbered list:

 1. first item
 2. second item
 3. third item

Note again how the actual text starts at 4 columns in (4 characters
from the left side). Here's a code sample:

    # Let me re-iterate ...
    for i in 1 .. 10 { do-something(i) }

As you probably guessed, indented 4 spaces. By the way, instead of
indenting the block, you can use delimited blocks, if you like:
\end{lstlisting}

define foobar() \{ print ``Welcome to flavor country!''; \}
\textasciitilde\textasciitilde\textasciitilde{}

(which makes copying \& pasting easier). You can optionally mark the
delimited block for Pandoc to syntax highlight it:

\begin{lstlisting}[language=Python]
import time
# Quick, count to ten!
for i in range(10):
    # (but not *too* quick)
    time.sleep(0.5)
    print(i)
\end{lstlisting}

\hypertarget{an-h3-header}{%
\subsubsection{An h3 header}\label{an-h3-header}}

Now a nested list:

\begin{enumerate}
\def\labelenumi{\arabic{enumi}.}
\item
  First, get these ingredients:

  \begin{itemize}
  \tightlist
  \item
    carrots
  \item
    celery
  \item
    lentils
  \end{itemize}
\item
  Boil some water.
\item
  Dump everything in the pot and follow this algorithm:

\begin{lstlisting}
find wooden spoon
uncover pot
stir
cover pot
balance wooden spoon precariously on pot handle
wait 10 minutes
goto first step (or shut off burner when done)
\end{lstlisting}

  Do not bump wooden spoon or it will fall.
\end{enumerate}

Notice again how text always lines up on 4-space indents (including that
last line which continues item 3 above).

Here's a link to \href{http://foo.bar}{a website}, to a
\href{local-doc.html}{local doc}, and to a
\protect\hyperlink{an-h2-header}{section heading in the current doc}.
Here's a footnote \footnote{Some footnote text.}.

Tables can look like this:

\begin{longtable}[]{@{}lrll@{}}
\caption{Shoes sizes, materials, and colors.}\tabularnewline
\toprule
Name & Size & Material & Color\tabularnewline
\midrule
\endfirsthead
\toprule
Name & Size & Material & Color\tabularnewline
\midrule
\endhead
All Business & 9 & leather & brown\tabularnewline
Roundabout & 10 & hemp canvas & natural\tabularnewline
Cinderella & 11 & glass & transparent\tabularnewline
\bottomrule
\end{longtable}

(The above is the caption for the table.) Pandoc also supports
multi-line tables:

\begin{longtable}[]{@{}ll@{}}
\toprule
\begin{minipage}[b]{0.13\columnwidth}\raggedright
Keyword\strut
\end{minipage} & \begin{minipage}[b]{0.31\columnwidth}\raggedright
Text\strut
\end{minipage}\tabularnewline
\midrule
\endhead
\begin{minipage}[t]{0.13\columnwidth}\raggedright
red\strut
\end{minipage} & \begin{minipage}[t]{0.31\columnwidth}\raggedright
Sunsets, apples, and other red or reddish things.\strut
\end{minipage}\tabularnewline
\begin{minipage}[t]{0.13\columnwidth}\raggedright
green\strut
\end{minipage} & \begin{minipage}[t]{0.31\columnwidth}\raggedright
Leaves, grass, frogs and other things it's not easy being.\strut
\end{minipage}\tabularnewline
\bottomrule
\end{longtable}

A horizontal rule follows.

\begin{center}\rule{0.5\linewidth}{\linethickness}\end{center}

Here's a definition list:

\begin{description}
\tightlist
\item[apples]
Good for making applesauce.
\item[oranges]
Citrus!
\item[tomatoes]
There's no ``e'' in tomatoe.
\end{description}

Again, text is indented 4 spaces. (Put a blank line between each term
and its definition to spread things out more.)

Here's a ``line block'' (note how whitespace is honored):

Line one\\
\hspace*{0.333em}\hspace*{0.333em}Line too\\
Line tree

and images can be specified like so:

\begin{figure}
\centering
\includegraphics{example-image.jpg}
\caption{example image}
\end{figure}

Inline math equation: \(\omega = d\phi / dt\). Display math should get
its own line like so:

\[I = \int \rho R^{2} dV\]

And note that you can backslash-escape any punctuation characters which
you wish to be displayed literally, ex.: `foo`, *bar*, etc.

\begin{center}\rule{0.5\linewidth}{\linethickness}\end{center}

And also, we may cite others scientific papers/book/whatever.

It's recommended to use as \passthrough{\lstinline!pandoc\_extra\_args!}
this: ``--filter=pandoc-citeproc''

Citing The Wiki Brazil \textcite{wiki:Brazil}, Citing The Wiki
Plagiarism \autocite{wiki:Plagiarism} and Citing the Deusch Language
Wiki \autocite*{wiki:Deutsch}. \_\_\_


\part*{Dummy project \\ \bigskip\bigskip\bigskip\bigskip\bigskip\normalfont\large{bertolino}}
\addcontentsline{toc}{part}{Dummy project}

\def \logbuchprojectmark {Dummy project}

\chapter*{\LARGE{Quick markdown example}}

\addcontentsline{toc}{chapter}{$\circ$ Quick markdown example}
\setcounter{section}{0}

\def \logbuchsubjectmark {Quick markdown example}
\def \logbuchsubjectdatemark {09.03.2019 02:40}
\def \logbuchsubjectauthmark {\textit{bertolino}}
\thispagestyle{eisvogel-header-footer}
\textbf{The content below was copied from
http://www.unexpected-vortices.com/sw/rippledoc/quick-markdown-example.html
at 2019-03-09 and has some modifications.}

\hypertarget{an-h1-header}{%
\section{An h1 header}\label{an-h1-header}}

Paragraphs are separated by a blank line.

2nd paragraph. \emph{Italic}, \textbf{bold}, and
\passthrough{\lstinline!monospace!}. Itemized lists look like:

\begin{itemize}
\tightlist
\item
  this one
\item
  that one
\item
  the other one
\end{itemize}

Note that --- not considering the asterisk --- the actual text content
starts at 4-columns in.

\begin{quote}
Block quotes are written like so.

They can span multiple paragraphs, if you like.
\end{quote}

Use 3 dashes for an em-dash. Use 2 dashes for ranges (ex., ``it's all in
chapters 12--14''). Three dots \ldots{} will be converted to an
ellipsis. Unicode is supported. \textbf{\emph{It depends on the output
format. You may face problems while converting an unicode character to
LaTeX. For example, if you use the ``+emoji'' extension, pandoc will
generate unicode.}}

\hypertarget{an-h2-header}{%
\subsection{An h2 header}\label{an-h2-header}}

Here's a numbered list:

\begin{enumerate}
\def\labelenumi{\arabic{enumi}.}
\tightlist
\item
  first item
\item
  second item
\item
  third item
\end{enumerate}

Note again how the actual text starts at 4 columns in (4 characters from
the left side). Here's a code sample:

\begin{lstlisting}
# Let me re-iterate ...
for i in 1 .. 10 { do-something(i) }
\end{lstlisting}

As you probably guessed, indented 4 spaces. By the way, instead of
indenting the block, you can use delimited blocks, if you like:

\begin{lstlisting}
define foobar() {
    print "Welcome to flavor country!";
An h1 header
============

Paragraphs are separated by a blank line.

2nd paragraph. *Italic*, **bold**, and `monospace`. Itemized lists
look like:

  * this one
  * that one
  * the other one

Note that --- not considering the asterisk --- the actual text
content starts at 4-columns in.

> Block quotes are
> written like so.
>
> They can span multiple paragraphs,
> if you like.

Use 3 dashes for an em-dash. Use 2 dashes for ranges (ex., "it's all
in chapters 12--14"). Three dots ... will be converted to an ellipsis.
Unicode is supported.



An h2 header
------------

Here's a numbered list:

 1. first item
 2. second item
 3. third item

Note again how the actual text starts at 4 columns in (4 characters
from the left side). Here's a code sample:

    # Let me re-iterate ...
    for i in 1 .. 10 { do-something(i) }

As you probably guessed, indented 4 spaces. By the way, instead of
indenting the block, you can use delimited blocks, if you like:
\end{lstlisting}

define foobar() \{ print ``Welcome to flavor country!''; \}
\textasciitilde\textasciitilde\textasciitilde{}

(which makes copying \& pasting easier). You can optionally mark the
delimited block for Pandoc to syntax highlight it:

\begin{lstlisting}[language=Python]
import time
# Quick, count to ten!
for i in range(10):
    # (but not *too* quick)
    time.sleep(0.5)
    print(i)
\end{lstlisting}

\hypertarget{an-h3-header}{%
\subsubsection{An h3 header}\label{an-h3-header}}

Now a nested list:

\begin{enumerate}
\def\labelenumi{\arabic{enumi}.}
\item
  First, get these ingredients:

  \begin{itemize}
  \tightlist
  \item
    carrots
  \item
    celery
  \item
    lentils
  \end{itemize}
\item
  Boil some water.
\item
  Dump everything in the pot and follow this algorithm:

\begin{lstlisting}
find wooden spoon
uncover pot
stir
cover pot
balance wooden spoon precariously on pot handle
wait 10 minutes
goto first step (or shut off burner when done)
\end{lstlisting}

  Do not bump wooden spoon or it will fall.
\end{enumerate}

Notice again how text always lines up on 4-space indents (including that
last line which continues item 3 above).

Here's a link to \href{http://foo.bar}{a website}, to a
\href{local-doc.html}{local doc}, and to a
\protect\hyperlink{an-h2-header}{section heading in the current doc}.
Here's a footnote \footnote{Some footnote text.}.

Tables can look like this:

\begin{longtable}[]{@{}lrll@{}}
\caption{Shoes sizes, materials, and colors.}\tabularnewline
\toprule
Name & Size & Material & Color\tabularnewline
\midrule
\endfirsthead
\toprule
Name & Size & Material & Color\tabularnewline
\midrule
\endhead
All Business & 9 & leather & brown\tabularnewline
Roundabout & 10 & hemp canvas & natural\tabularnewline
Cinderella & 11 & glass & transparent\tabularnewline
\bottomrule
\end{longtable}

(The above is the caption for the table.) Pandoc also supports
multi-line tables:

\begin{longtable}[]{@{}ll@{}}
\toprule
\begin{minipage}[b]{0.13\columnwidth}\raggedright
Keyword\strut
\end{minipage} & \begin{minipage}[b]{0.31\columnwidth}\raggedright
Text\strut
\end{minipage}\tabularnewline
\midrule
\endhead
\begin{minipage}[t]{0.13\columnwidth}\raggedright
red\strut
\end{minipage} & \begin{minipage}[t]{0.31\columnwidth}\raggedright
Sunsets, apples, and other red or reddish things.\strut
\end{minipage}\tabularnewline
\begin{minipage}[t]{0.13\columnwidth}\raggedright
green\strut
\end{minipage} & \begin{minipage}[t]{0.31\columnwidth}\raggedright
Leaves, grass, frogs and other things it's not easy being.\strut
\end{minipage}\tabularnewline
\bottomrule
\end{longtable}

A horizontal rule follows.

\begin{center}\rule{0.5\linewidth}{\linethickness}\end{center}

Here's a definition list:

\begin{description}
\tightlist
\item[apples]
Good for making applesauce.
\item[oranges]
Citrus!
\item[tomatoes]
There's no ``e'' in tomatoe.
\end{description}

Again, text is indented 4 spaces. (Put a blank line between each term
and its definition to spread things out more.)

Here's a ``line block'' (note how whitespace is honored):

Line one\\
\hspace*{0.333em}\hspace*{0.333em}Line too\\
Line tree

and images can be specified like so:

\begin{figure}
\centering
\includegraphics{example-image.jpg}
\caption{example image}
\end{figure}

Inline math equation: \(\omega = d\phi / dt\). Display math should get
its own line like so:

\[I = \int \rho R^{2} dV\]

And note that you can backslash-escape any punctuation characters which
you wish to be displayed literally, ex.: `foo`, *bar*, etc.



\printbibliography

\end{document}
